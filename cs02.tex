% Options for packages loaded elsewhere
\PassOptionsToPackage{unicode}{hyperref}
\PassOptionsToPackage{hyphens}{url}
%
\documentclass[
]{article}
\usepackage{amsmath,amssymb}
\usepackage{iftex}
\ifPDFTeX
  \usepackage[T1]{fontenc}
  \usepackage[utf8]{inputenc}
  \usepackage{textcomp} % provide euro and other symbols
\else % if luatex or xetex
  \usepackage{unicode-math} % this also loads fontspec
  \defaultfontfeatures{Scale=MatchLowercase}
  \defaultfontfeatures[\rmfamily]{Ligatures=TeX,Scale=1}
\fi
\usepackage{lmodern}
\ifPDFTeX\else
  % xetex/luatex font selection
\fi
% Use upquote if available, for straight quotes in verbatim environments
\IfFileExists{upquote.sty}{\usepackage{upquote}}{}
\IfFileExists{microtype.sty}{% use microtype if available
  \usepackage[]{microtype}
  \UseMicrotypeSet[protrusion]{basicmath} % disable protrusion for tt fonts
}{}
\makeatletter
\@ifundefined{KOMAClassName}{% if non-KOMA class
  \IfFileExists{parskip.sty}{%
    \usepackage{parskip}
  }{% else
    \setlength{\parindent}{0pt}
    \setlength{\parskip}{6pt plus 2pt minus 1pt}}
}{% if KOMA class
  \KOMAoptions{parskip=half}}
\makeatother
\usepackage{xcolor}
\usepackage[margin=1in]{geometry}
\usepackage{color}
\usepackage{fancyvrb}
\newcommand{\VerbBar}{|}
\newcommand{\VERB}{\Verb[commandchars=\\\{\}]}
\DefineVerbatimEnvironment{Highlighting}{Verbatim}{commandchars=\\\{\}}
% Add ',fontsize=\small' for more characters per line
\usepackage{framed}
\definecolor{shadecolor}{RGB}{248,248,248}
\newenvironment{Shaded}{\begin{snugshade}}{\end{snugshade}}
\newcommand{\AlertTok}[1]{\textcolor[rgb]{0.94,0.16,0.16}{#1}}
\newcommand{\AnnotationTok}[1]{\textcolor[rgb]{0.56,0.35,0.01}{\textbf{\textit{#1}}}}
\newcommand{\AttributeTok}[1]{\textcolor[rgb]{0.13,0.29,0.53}{#1}}
\newcommand{\BaseNTok}[1]{\textcolor[rgb]{0.00,0.00,0.81}{#1}}
\newcommand{\BuiltInTok}[1]{#1}
\newcommand{\CharTok}[1]{\textcolor[rgb]{0.31,0.60,0.02}{#1}}
\newcommand{\CommentTok}[1]{\textcolor[rgb]{0.56,0.35,0.01}{\textit{#1}}}
\newcommand{\CommentVarTok}[1]{\textcolor[rgb]{0.56,0.35,0.01}{\textbf{\textit{#1}}}}
\newcommand{\ConstantTok}[1]{\textcolor[rgb]{0.56,0.35,0.01}{#1}}
\newcommand{\ControlFlowTok}[1]{\textcolor[rgb]{0.13,0.29,0.53}{\textbf{#1}}}
\newcommand{\DataTypeTok}[1]{\textcolor[rgb]{0.13,0.29,0.53}{#1}}
\newcommand{\DecValTok}[1]{\textcolor[rgb]{0.00,0.00,0.81}{#1}}
\newcommand{\DocumentationTok}[1]{\textcolor[rgb]{0.56,0.35,0.01}{\textbf{\textit{#1}}}}
\newcommand{\ErrorTok}[1]{\textcolor[rgb]{0.64,0.00,0.00}{\textbf{#1}}}
\newcommand{\ExtensionTok}[1]{#1}
\newcommand{\FloatTok}[1]{\textcolor[rgb]{0.00,0.00,0.81}{#1}}
\newcommand{\FunctionTok}[1]{\textcolor[rgb]{0.13,0.29,0.53}{\textbf{#1}}}
\newcommand{\ImportTok}[1]{#1}
\newcommand{\InformationTok}[1]{\textcolor[rgb]{0.56,0.35,0.01}{\textbf{\textit{#1}}}}
\newcommand{\KeywordTok}[1]{\textcolor[rgb]{0.13,0.29,0.53}{\textbf{#1}}}
\newcommand{\NormalTok}[1]{#1}
\newcommand{\OperatorTok}[1]{\textcolor[rgb]{0.81,0.36,0.00}{\textbf{#1}}}
\newcommand{\OtherTok}[1]{\textcolor[rgb]{0.56,0.35,0.01}{#1}}
\newcommand{\PreprocessorTok}[1]{\textcolor[rgb]{0.56,0.35,0.01}{\textit{#1}}}
\newcommand{\RegionMarkerTok}[1]{#1}
\newcommand{\SpecialCharTok}[1]{\textcolor[rgb]{0.81,0.36,0.00}{\textbf{#1}}}
\newcommand{\SpecialStringTok}[1]{\textcolor[rgb]{0.31,0.60,0.02}{#1}}
\newcommand{\StringTok}[1]{\textcolor[rgb]{0.31,0.60,0.02}{#1}}
\newcommand{\VariableTok}[1]{\textcolor[rgb]{0.00,0.00,0.00}{#1}}
\newcommand{\VerbatimStringTok}[1]{\textcolor[rgb]{0.31,0.60,0.02}{#1}}
\newcommand{\WarningTok}[1]{\textcolor[rgb]{0.56,0.35,0.01}{\textbf{\textit{#1}}}}
\usepackage{longtable,booktabs,array}
\usepackage{calc} % for calculating minipage widths
% Correct order of tables after \paragraph or \subparagraph
\usepackage{etoolbox}
\makeatletter
\patchcmd\longtable{\par}{\if@noskipsec\mbox{}\fi\par}{}{}
\makeatother
% Allow footnotes in longtable head/foot
\IfFileExists{footnotehyper.sty}{\usepackage{footnotehyper}}{\usepackage{footnote}}
\makesavenoteenv{longtable}
\usepackage{graphicx}
\makeatletter
\def\maxwidth{\ifdim\Gin@nat@width>\linewidth\linewidth\else\Gin@nat@width\fi}
\def\maxheight{\ifdim\Gin@nat@height>\textheight\textheight\else\Gin@nat@height\fi}
\makeatother
% Scale images if necessary, so that they will not overflow the page
% margins by default, and it is still possible to overwrite the defaults
% using explicit options in \includegraphics[width, height, ...]{}
\setkeys{Gin}{width=\maxwidth,height=\maxheight,keepaspectratio}
% Set default figure placement to htbp
\makeatletter
\def\fps@figure{htbp}
\makeatother
\setlength{\emergencystretch}{3em} % prevent overfull lines
\providecommand{\tightlist}{%
  \setlength{\itemsep}{0pt}\setlength{\parskip}{0pt}}
\setcounter{secnumdepth}{-\maxdimen} % remove section numbering
\ifLuaTeX
  \usepackage{selnolig}  % disable illegal ligatures
\fi
\usepackage{bookmark}
\IfFileExists{xurl.sty}{\usepackage{xurl}}{} % add URL line breaks if available
\urlstyle{same}
\hypersetup{
  pdftitle={CS02 - Predicting Annual Air Pollution},
  pdfauthor={Jiayi Xue, Nick Liu, Roxana Chen, Sehee Kim, Jorge Ramos},
  hidelinks,
  pdfcreator={LaTeX via pandoc}}

\title{CS02 - Predicting Annual Air Pollution}
\author{Jiayi Xue, Nick Liu, Roxana Chen, Sehee Kim, Jorge Ramos}
\date{}

\begin{document}
\maketitle

{
\setcounter{tocdepth}{2}
\tableofcontents
}
\subsection{Introduction}\label{introduction}

\subsection{Background}\label{background}

With health impacts such as asthma, lung function, low birthweight, and
viral infection being globally widespread, air pollution is at fifth
place for global risk factors by total number of deaths from all causes
in 2017 \footnote{\url{https://www.stateofglobalair.org/sites/default/files/soga_2019_report.pdf}}.
With particles as small as 0.43 micrometers reaching into the human
respiratory system, it is extremely difficult to catch visible effects
until it is too late. This awareness has been inconsistent throughout
the world, with some countries exceeding the World Health Organization
(WHO) guideline for healthy air while other countries such as Africa and
Asia displaying extremely poor air quality \footnote{\url{https://www.stateofglobalair.org/sites/default/files/soga_2019_fact_sheet.pdf}}.
Despite the fact that air quality has improved throughout time in the
United States, it is still a major health risk. With air pollution
monitors being sparsely spread throughout the contiguous United States,
this may lead to inaccurate readings in significant regions often
minimizing our knowledge on the relationship between air pollution and
its health impacts. To reduce this gap, data from these monitors along
with multiple external factors that manipulate air pollution
concentration (urbanization, population density, road length, emissions,
etc.,) is taken into account to make further analysis. Taking all these
factors into consideration, \textbf{with what accuracy can we predict US
annual average air pollution concentration?} Depending on this accuracy,
we can take efficient and practical measures to avoid prominent health
issues while raising awareness.

Extending, \textbf{can we find a more parsimonious model that we can
utilize for various situations such as not having all the variables?}
Occam's Razor states that with a model of multiple predictors, it is
best to minimize the amount of predictors in order to improve efficiency
while maintaining accuracy: a parsimonious model. With this principle
and using feature selection, our team found another model that was more
practical and efficient to use when predicting air pollution
concentration. Specifically, we can still find the concentration for
different areas even when we do not and cannot measure all the
predictors.

We were given 876 monitors in the US with 48 various observations,
explained by the table below. Note that although we are solely measuring
fine particulate matter (PM\_2.5), there are three other major types of
air pollutants - gaseous, dust, and biological - and three other sizes
of particulate pollution - coarse and large coarse. These are all types
of pollution that significantly impact our health and should frequently
be kept in mind.

\textbf{This is our data variables and their description}

\begin{longtable}[]{@{}
  >{\raggedright\arraybackslash}p{(\columnwidth - 2\tabcolsep) * \real{0.2805}}
  >{\raggedright\arraybackslash}p{(\columnwidth - 2\tabcolsep) * \real{0.7195}}@{}}
\toprule\noalign{}
\begin{minipage}[b]{\linewidth}\raggedright
Variable
\end{minipage} & \begin{minipage}[b]{\linewidth}\raggedright
Description
\end{minipage} \\
\midrule\noalign{}
\endhead
\bottomrule\noalign{}
\endlastfoot
id & monitor number (country number before the decimal, monitor number
after the decimal) \\
fips & federal information processing standard number for the country
where the monitor is located \\
lat & latitude of the monitor in degrees \\
lon & longitude of the monitor in degrees \\
state & state where the monitor is located \\
county & county where the monitor is located \\
city & city where the monitor is located \\
CMAQ & estimated values of air pollution from Community Multiscale Air
Quality, a monitoring system that uses chemistry and weather data to
predict the air pollution \\
zcta & Zip Code Tabulation Area where the monitor is located (2010) \\
zcta\_area & land area of the zip code area in meters squared (2010) \\
zcta\_pop & population in the zip code area (2010) \\
country\_area & land area of the county of the monitor in meters
squared \\
country\_pop & population of the county of the monitor \\
imp\_a(radius around the monitor) & impervious surface measure such as
roads, concrete, parking lots, and buildings as a measure of
development \\
Log\_dist\_to\_prisec & log (natural log) distance to a primary or
secondary road from the monitor \\
log\_pri\_length\_(radius around the monitor & count of primary road
length in meters in a circle in various radius \\
log\_prisec\_length\_(radius around the monitor) & count of primary and
secondary road length in meters in a circle in various radius \\
log\_nei\_2008\_pm25\_sum\_(radius of distance around the monitor) &
tons of emissions from major sources database (annual data) sum of all
sources within a circle in various radius meters of distance around the
monitor \\
popdens\_zcta & population density (number of people per kilometer
squared area of zcta) \\
nohs & percentage of people in zcta area where the monitor is that do
not have a high school degree \\
somehs & percentage of people in zcta area where the monitor whose
highest formal educational attainment was some high school education \\
hs & percentage of people in zcta area where the monitor whose highest
formal educational attainment was completing a high school degree \\
somecollege & percentage of people in zcta area where the monitor whose
highest formal educational attainment was completing some college
education \\
associate & percentage of people in zcta area where the monitor whose
highest formal educational attainment was completing an associate
degree \\
bachelor & percentage of people in zcta area where the monitor whose
highest formal educational attainment was a bachelor's degree \\
grad & percentage of people in zcta area where the monitor whose highest
formal educational attainment was a graduate degree \\
pov & percentage of people in zcta area where the monitor is that lived
in poverty in 2008 \\
hs\_orless & percentage of people in zcta area where the monitor whose
highest formal educational attainment was a high school degree or less
(sum of nohs, somehs, and hs) \\
urc2006 \& urc2013 & urban-rural classification of the county where the
monitor is located (1 - totally urban and 6 - completely rural) \\
aod & Aerosol Optical Depth measurement from NASA satellite used as a
proxy of particulate pollution \\
\end{longtable}

\subsubsection{Load packages}\label{load-packages}

\begin{Shaded}
\begin{Highlighting}[]
\FunctionTok{library}\NormalTok{(tidyverse)}
\FunctionTok{library}\NormalTok{(tidymodels)}
\FunctionTok{library}\NormalTok{(sf)}
\FunctionTok{library}\NormalTok{(maps)}
\FunctionTok{library}\NormalTok{(rnaturalearth)}
\FunctionTok{library}\NormalTok{(patchwork)}
\end{Highlighting}
\end{Shaded}

\subsection{The Data}\label{the-data}

\subsubsection{Data Import}\label{data-import}

\begin{Shaded}
\begin{Highlighting}[]
\NormalTok{pm }\OtherTok{\textless{}{-}} \FunctionTok{read\_csv}\NormalTok{(}\StringTok{"pm25\_data.csv"}\NormalTok{)}
\end{Highlighting}
\end{Shaded}

\subsubsection{Data Wrangling}\label{data-wrangling}

In our data wrangling, we performed factoring, data splitting, and cross
validation. We factored \texttt{id}, \texttt{fips}, \texttt{zcta} to
ensure they are treated as categorical values in the model. We also
simplified the city feature to be binary, distinguishing monitors in
cities and out of cities.

\begin{Shaded}
\begin{Highlighting}[]
\CommentTok{\# Converting to factors}
\NormalTok{pm }\OtherTok{\textless{}{-}}\NormalTok{ pm }\SpecialCharTok{|\textgreater{}}
  \FunctionTok{mutate}\NormalTok{(}\FunctionTok{across}\NormalTok{(}\FunctionTok{c}\NormalTok{(id, fips, zcta), as.factor)) }

\CommentTok{\# Making it clear which monitor is in a city or not in a city}
\NormalTok{pm }\OtherTok{\textless{}{-}}\NormalTok{ pm }\SpecialCharTok{|\textgreater{}}
  \FunctionTok{mutate}\NormalTok{(}\AttributeTok{city =} \FunctionTok{case\_when}\NormalTok{(city }\SpecialCharTok{==} \StringTok{"Not in a city"} \SpecialCharTok{\textasciitilde{}} \StringTok{"Not in a city"}\NormalTok{,}
\NormalTok{                          city }\SpecialCharTok{!=} \StringTok{"Not in a city"} \SpecialCharTok{\textasciitilde{}} \StringTok{"In a city"}\NormalTok{))}
\end{Highlighting}
\end{Shaded}

Data splitting is critical in model prediction, we split data into
training and testing categories. We use the training set to train the
model by finding patterns in the data and we use the testing set to
evaluate the model performance. The goal of data splitting is to test
the model on data it has never seen during training to check for
overfitting and assess its performance. We initiate the total dataset
(876), split ⅔ of it into the training set (584) and ⅓ of it into the
testing set (292). To enhance our model's generalizability, we
implemented 4-fold cross-validation using the vfold\_cv function from
the rsample package. The 4-fold cross-validation involves partitioning
the training data set into 4 equal folds. The model is trained on three
folds and validated on the fourth, cycling through all folds as the
validation set. This ensures that the model is evaluated on multiple
subsets of the training data and provides a more robust and reliable
model. Cross-validation also helps mitigate the risk of overfitting and
increases its generalizability.

\begin{Shaded}
\begin{Highlighting}[]
\CommentTok{\# setting a seed}
\FunctionTok{set.seed}\NormalTok{(}\DecValTok{1234}\NormalTok{)}

\CommentTok{\# Splitting data into train/testing split}

\NormalTok{pm\_split }\OtherTok{\textless{}{-}}\NormalTok{ rsample}\SpecialCharTok{::}\FunctionTok{initial\_split}\NormalTok{(}\AttributeTok{data =}\NormalTok{ pm, }\AttributeTok{prop =} \DecValTok{2}\SpecialCharTok{/}\DecValTok{3}\NormalTok{)}
\NormalTok{pm\_split}
\end{Highlighting}
\end{Shaded}

\begin{verbatim}
## <Training/Testing/Total>
## <584/292/876>
\end{verbatim}

\begin{Shaded}
\begin{Highlighting}[]
\NormalTok{train\_pm }\OtherTok{\textless{}{-}}\NormalTok{ rsample}\SpecialCharTok{::}\FunctionTok{training}\NormalTok{(pm\_split)}
\NormalTok{test\_pm }\OtherTok{\textless{}{-}}\NormalTok{ rsample}\SpecialCharTok{::}\FunctionTok{testing}\NormalTok{(pm\_split)}
 
\CommentTok{\# Performing cross validation}
\NormalTok{vfold\_pm }\OtherTok{\textless{}{-}}\NormalTok{ rsample}\SpecialCharTok{::}\FunctionTok{vfold\_cv}\NormalTok{(}\AttributeTok{data =}\NormalTok{ train\_pm, }\AttributeTok{v =} \DecValTok{4}\NormalTok{)}
\NormalTok{vfold\_pm}
\end{Highlighting}
\end{Shaded}

\begin{verbatim}
## #  4-fold cross-validation 
## # A tibble: 4 x 2
##   splits            id   
##   <list>            <chr>
## 1 <split [438/146]> Fold1
## 2 <split [438/146]> Fold2
## 3 <split [438/146]> Fold3
## 4 <split [438/146]> Fold4
\end{verbatim}

\subsection{Analysis}\label{analysis}

\subsubsection{Exploratory Data Analysis
(EDA)}\label{exploratory-data-analysis-eda}

\begin{Shaded}
\begin{Highlighting}[]
\CommentTok{\# loading the world map}
\NormalTok{world }\OtherTok{\textless{}{-}} \FunctionTok{ne\_countries}\NormalTok{(}\AttributeTok{scale =} \StringTok{"medium"}\NormalTok{, }\AttributeTok{returnclass =} \StringTok{"sf"}\NormalTok{)}

\CommentTok{\# plotting the monitor locations onto a world map}
\FunctionTok{ggplot}\NormalTok{(}\AttributeTok{data =}\NormalTok{ world) }\SpecialCharTok{+}
    \FunctionTok{geom\_sf}\NormalTok{() }\SpecialCharTok{+}
    \FunctionTok{coord\_sf}\NormalTok{(}\AttributeTok{xlim =} \FunctionTok{c}\NormalTok{(}\SpecialCharTok{{-}}\DecValTok{125}\NormalTok{, }\SpecialCharTok{{-}}\DecValTok{66}\NormalTok{), }\AttributeTok{ylim =} \FunctionTok{c}\NormalTok{(}\FloatTok{24.5}\NormalTok{, }\DecValTok{50}\NormalTok{), }
             \AttributeTok{expand =} \ConstantTok{FALSE}\NormalTok{)}\SpecialCharTok{+}
    \FunctionTok{geom\_point}\NormalTok{(}\AttributeTok{data =}\NormalTok{ pm, }\FunctionTok{aes}\NormalTok{(}\AttributeTok{x =}\NormalTok{ lon, }\AttributeTok{y =}\NormalTok{ lat), }\AttributeTok{size =} \DecValTok{2}\NormalTok{, }
               \AttributeTok{shape =} \DecValTok{23}\NormalTok{, }\AttributeTok{fill =} \StringTok{"darkred"}\NormalTok{) }\SpecialCharTok{+} 
    \FunctionTok{labs}\NormalTok{(}\AttributeTok{title =} \StringTok{"Different Concentrations of Monitors Across the US"}\NormalTok{, }\AttributeTok{x =} \StringTok{"Latitude"}\NormalTok{, }\AttributeTok{y =} \StringTok{"Longitude"}\NormalTok{)}
\end{Highlighting}
\end{Shaded}

\includegraphics{cs02_files/figure-latex/EDA-world-1.pdf}

Exploring the data, we first mapped out all of the monitors onto a map
of the US. From this map visualization, we are able to see how the
amount of monitors is not a consistent spread across the US, but rather
many monitors tend to be bunched up in specific areas while very sparse
in others. Specifically, monitors tend to have large concentrations
along the coast, possibly due to having a higher population along
coastlines.

\begin{Shaded}
\begin{Highlighting}[]
\CommentTok{\# counting the monitors in each state, listed from highest to lowest}
\FunctionTok{count}\NormalTok{(pm, state) }\SpecialCharTok{|\textgreater{}}
  \FunctionTok{arrange}\NormalTok{(}\FunctionTok{desc}\NormalTok{(n))}
\end{Highlighting}
\end{Shaded}

\begin{verbatim}
## # A tibble: 49 x 2
##    state              n
##    <chr>          <int>
##  1 California        85
##  2 Ohio              44
##  3 Illinois          38
##  4 Indiana           36
##  5 North Carolina    35
##  6 Pennsylvania      32
##  7 Michigan          30
##  8 Florida           29
##  9 Georgia           28
## 10 Texas             27
## # i 39 more rows
\end{verbatim}

Looking further, specifically at the amount of monitors in each state,
we can see the concentration of monitors is not only different across
the country, but also different across states. For example, California
has the highest amount of monitors at 85, which is almost double the
amount of monitors than there is for the state with the second highest
amount of monitors, being Ohio with 44. With this, we can see the
inconsistency across states, and know that the varied spread of monitors
across the US as seen in the previous world map plot is not necessarily
dictated by state boundaries.

\begin{Shaded}
\begin{Highlighting}[]
\CommentTok{\# creating a correlation plot for all the variables}
\NormalTok{pm\_cor }\OtherTok{\textless{}{-}} \FunctionTok{cor}\NormalTok{(pm }\SpecialCharTok{|\textgreater{}}\NormalTok{ dplyr}\SpecialCharTok{::}\FunctionTok{select\_if}\NormalTok{(is.numeric))}
\NormalTok{corrplot}\SpecialCharTok{::}\FunctionTok{corrplot}\NormalTok{(pm\_cor, }\AttributeTok{tl.cex =} \FloatTok{0.5}\NormalTok{)}
\end{Highlighting}
\end{Shaded}

\includegraphics{cs02_files/figure-latex/EDA-corr-1.pdf}

The correlation plot above shows both strength and direction of
correlation between all of the variables in the dataset. For example, a
deep blue color at the intersection of two variables indicates a strong,
positive correlation, while a deep red color at the intersection of two
variables shows a strong, negative correlation.

As the plot shows, each group of variables (i.e.~log\_pri\_length\_5000,
log\_pri\_length\_10000, etc. would be an example of variables in the
same group) are highly positively correlated with each other, which is
to be expected because the variables are measuring the same parameter,
just on different scopes (ex. log\_pri\_length\_5000 and
log\_pri\_length\_10000 are both measuring primary road length, but for
different radii). A few more notable observations from the plot are
that: - CMAQ, a predictive measure of air pollution, is the most highly
(and positively) correlated variable with value, the variable which
reflects the actual measure of air pollution - CMAQ is more highly
correlated with value than aod, another predictive air pollution
variable - CMAQ is relatively strongly negatively correlated with urc
measurements, which are a classification scale of the urban/rural level
of an area.

\begin{Shaded}
\begin{Highlighting}[]
\CommentTok{\# plotting the relationship between CMAQ \& urc\_2013}
\FunctionTok{ggplot}\NormalTok{(pm, }\FunctionTok{aes}\NormalTok{(}\AttributeTok{x =} \FunctionTok{factor}\NormalTok{(urc2013), }\AttributeTok{y =}\NormalTok{ CMAQ, }\AttributeTok{fill =}\NormalTok{ urc2013)) }\SpecialCharTok{+} 
  \FunctionTok{geom\_boxplot}\NormalTok{() }\SpecialCharTok{+}
  \FunctionTok{labs}\NormalTok{(}\AttributeTok{x =} \StringTok{"URC 2013"}\NormalTok{)}
\end{Highlighting}
\end{Shaded}

\includegraphics{cs02_files/figure-latex/EDA-1.pdf}

Based on the previous correlation seen, we looked further into the
relationship between CMAQ and the urc\_2013 variables and found that the
more rural an area is (on the urban-rural classification scale, 1 is
more urban and 6 is more rural), the lower the predicted value of air
pollution in the area (lower CMAQ value). The opposite is also true, as
seen in the graph above; the more urban an area is, the higher the
predicted value of air pollution (higer CMAQ value).

\subsubsection{Model}\label{model}

After splitting the data into testing and training sets, we built the
model using tidymodels. The steps start with creating ``recipe'' and
assigning variable roles. Then, we specify the model, engine, and mode.
Next, we fit workflow and get predictions. Finally, we use the
predictions to find performance metrics.

\begin{Shaded}
\begin{Highlighting}[]
\CommentTok{\# Recipe with all predictors}
\NormalTok{RF\_rec }\OtherTok{\textless{}{-}} \FunctionTok{recipe}\NormalTok{(train\_pm) }\SpecialCharTok{|\textgreater{}}
    \CommentTok{\# assign roles for the columns}
    \FunctionTok{update\_role}\NormalTok{(}\FunctionTok{everything}\NormalTok{(), }\AttributeTok{new\_role =} \StringTok{"predictor"}\NormalTok{)}\SpecialCharTok{|\textgreater{}}
    \FunctionTok{update\_role}\NormalTok{(value, }\AttributeTok{new\_role =} \StringTok{"outcome"}\NormalTok{)}\SpecialCharTok{|\textgreater{}}
    \FunctionTok{update\_role}\NormalTok{(id, }\AttributeTok{new\_role =} \StringTok{"id variable"}\NormalTok{) }\SpecialCharTok{|\textgreater{}}
    \FunctionTok{update\_role}\NormalTok{(}\StringTok{"fips"}\NormalTok{, }\AttributeTok{new\_role =} \StringTok{"county id"}\NormalTok{) }\SpecialCharTok{|\textgreater{}}
    
    \CommentTok{\# step\_novel assigns a previously unseen factor level to a new value }
    \FunctionTok{step\_novel}\NormalTok{(}\StringTok{"state"}\NormalTok{) }\SpecialCharTok{|\textgreater{}}

    \CommentTok{\# converts string to factors}
    \FunctionTok{step\_string2factor}\NormalTok{(}\StringTok{"state"}\NormalTok{, }\StringTok{"county"}\NormalTok{, }\StringTok{"city"}\NormalTok{) }\SpecialCharTok{|\textgreater{}}
  
    \CommentTok{\# removing redundancy}
    \FunctionTok{step\_rm}\NormalTok{(}\StringTok{"county"}\NormalTok{) }\SpecialCharTok{|\textgreater{}}
    \FunctionTok{step\_rm}\NormalTok{(}\StringTok{"zcta"}\NormalTok{) }\SpecialCharTok{|\textgreater{}}
    \CommentTok{\# removing highly correlated variables}
    \FunctionTok{step\_corr}\NormalTok{(}\FunctionTok{all\_numeric}\NormalTok{())}\SpecialCharTok{|\textgreater{}}
    \CommentTok{\# removing near zero variance predictors}
    \FunctionTok{step\_nzv}\NormalTok{(}\FunctionTok{all\_numeric}\NormalTok{())}
\end{Highlighting}
\end{Shaded}

The purpose of recipe() function is to provide a standardized structure
to ensure that the data is in the correct format for modeling. The
recipe, RF\_rec, defines a pre-processing pipeline for the training
dataset to prepare for the model. In the recipe, we initially assign all
variables in the dataset as predictors. Then, we override the role for
the column value as the ``outcome'', which is our target predicting
variable. \texttt{id} is specified as an ``id variable,'' which means it
won't be used as a predictor or outcome. The ``fips'' column was
assigned the new role of ``county id'' which is used for grouping but
not for predicting.

After creating the recipe(), several step\_*() functions are added to
define specific pre-processing steps. step\_novel() is used for the
``state'' variable to get all cross-validation folds to work by ensuring
that any unseen level in the ``state'' variables which wasn't observed
during training will be assigned a placeholder value to prevent errors
such as overfitting. We then convert the string columns ``state'',
``county'', and ``city'' to factors, making them suitable for
categorical handling in modeling. Next, we removed columns ``county''
and ``zcta'' from the dataset as they are redundant, removed numeric
variables that are highly correlated with each other, and removed
numeric predictors with near-zero variance whose values barely change
across observations, which contribute very little to the model
prediction.

\begin{Shaded}
\begin{Highlighting}[]
\CommentTok{\# untuned model to find most important parameters}
\NormalTok{RF\_PM\_model }\OtherTok{\textless{}{-}}\NormalTok{ parsnip}\SpecialCharTok{::}\FunctionTok{rand\_forest}\NormalTok{(}\AttributeTok{mtry =} \DecValTok{10}\NormalTok{, }\AttributeTok{min\_n =} \DecValTok{3}\NormalTok{) }\SpecialCharTok{|\textgreater{}} 
  \FunctionTok{set\_engine}\NormalTok{(}\StringTok{"randomForest"}\NormalTok{) }\SpecialCharTok{|\textgreater{}}
  \FunctionTok{set\_mode}\NormalTok{(}\StringTok{"regression"}\NormalTok{)}
\end{Highlighting}
\end{Shaded}

A tuned random forest model was then initialized to improve predictive
performance. Two hyperparameters, mtry and min\_n, were included for
tuning: mtry specifies the number of features randomly selected at each
split in a tree, while min\_n defines the minimum number of observations
required in a node to allow further splitting. Rather than manually
testing different values, the tune() function was used to automate the
hyperparameter optimization process. The model engine was set to
``randomForest,'' specifying the algorithm for building the random
forest, and set\_mode(``regression'') was used to indicate that the
target variable is continuous (e.g.~air pollution levels).

\begin{Shaded}
\begin{Highlighting}[]
\CommentTok{\# untuned workflow}
\NormalTok{RF\_wflow }\OtherTok{\textless{}{-}}\NormalTok{ workflows}\SpecialCharTok{::}\FunctionTok{workflow}\NormalTok{() }\SpecialCharTok{|\textgreater{}}
\NormalTok{  workflows}\SpecialCharTok{::}\FunctionTok{add\_recipe}\NormalTok{(RF\_rec) }\SpecialCharTok{|\textgreater{}}
\NormalTok{  workflows}\SpecialCharTok{::}\FunctionTok{add\_model}\NormalTok{(RF\_PM\_model)}

\CommentTok{\# fit with training set}
\NormalTok{RF\_wflow\_fit }\OtherTok{\textless{}{-}}\NormalTok{ parsnip}\SpecialCharTok{::}\FunctionTok{fit}\NormalTok{(RF\_wflow, }\AttributeTok{data =}\NormalTok{ train\_pm)}

\CommentTok{\# initializing a tuned random forest model}
\NormalTok{tune\_RF\_model }\OtherTok{\textless{}{-}} \FunctionTok{rand\_forest}\NormalTok{(}\AttributeTok{mtry =} \FunctionTok{tune}\NormalTok{(), }\AttributeTok{min\_n =} \FunctionTok{tune}\NormalTok{()) }\SpecialCharTok{|\textgreater{}}
  \FunctionTok{set\_engine}\NormalTok{(}\StringTok{"randomForest"}\NormalTok{) }\SpecialCharTok{|\textgreater{}}
  \FunctionTok{set\_mode}\NormalTok{(}\StringTok{"regression"}\NormalTok{)}

\CommentTok{\# create workflow}
\NormalTok{RF\_tune\_wflow }\OtherTok{\textless{}{-}}\NormalTok{ workflows}\SpecialCharTok{::}\FunctionTok{workflow}\NormalTok{() }\SpecialCharTok{|\textgreater{}}
\NormalTok{  workflows}\SpecialCharTok{::}\FunctionTok{add\_recipe}\NormalTok{(RF\_rec) }\SpecialCharTok{|\textgreater{}}
\NormalTok{  workflows}\SpecialCharTok{::}\FunctionTok{add\_model}\NormalTok{(tune\_RF\_model)}

\CommentTok{\# find number of cores}
\NormalTok{n\_cores }\OtherTok{\textless{}{-}}\NormalTok{ parallel}\SpecialCharTok{::}\FunctionTok{detectCores}\NormalTok{()}

\CommentTok{\# enable parallelism}
\NormalTok{doParallel}\SpecialCharTok{::}\FunctionTok{registerDoParallel}\NormalTok{(}\AttributeTok{cores =}\NormalTok{ n\_cores)}
\end{Highlighting}
\end{Shaded}

A workflow was created to combine pre-processing steps and modeling into
a single object, simplifying the machine learning pipeline. The recipe
(RF\_rec) created earlier is now added to the workflow, ensuring that
the data is pre-processed consistently before model training. Next,
tune\_RF\_model is added to the workflow to enable training.

\begin{Shaded}
\begin{Highlighting}[]
\CommentTok{\# perform hyper parameter tuning for random forest model using grid search across re samples}
\FunctionTok{set.seed}\NormalTok{(}\DecValTok{123}\NormalTok{)}
\NormalTok{tune\_RF\_results }\OtherTok{\textless{}{-}} \FunctionTok{tune\_grid}\NormalTok{(}\AttributeTok{object =}\NormalTok{ RF\_tune\_wflow, }\AttributeTok{resamples =}\NormalTok{ vfold\_pm, }\AttributeTok{grid =} \DecValTok{20}\NormalTok{)}

\CommentTok{\# results}
\NormalTok{tune\_RF\_results }\SpecialCharTok{|\textgreater{}}
  \FunctionTok{collect\_metrics}\NormalTok{()}
\end{Highlighting}
\end{Shaded}

\begin{verbatim}
## # A tibble: 40 x 8
##     mtry min_n .metric .estimator  mean     n std_err .config              
##    <int> <int> <chr>   <chr>      <dbl> <int>   <dbl> <chr>                
##  1     1    14 rmse    standard   2.00      4  0.127  Preprocessor1_Model01
##  2     1    14 rsq     standard   0.461     4  0.0489 Preprocessor1_Model01
##  3     2    24 rmse    standard   1.89      4  0.137  Preprocessor1_Model02
##  4     2    24 rsq     standard   0.517     4  0.0509 Preprocessor1_Model02
##  5     4    34 rmse    standard   1.81      4  0.144  Preprocessor1_Model03
##  6     4    34 rsq     standard   0.551     4  0.0521 Preprocessor1_Model03
##  7     6     6 rmse    standard   1.69      4  0.148  Preprocessor1_Model04
##  8     6     6 rsq     standard   0.599     4  0.0538 Preprocessor1_Model04
##  9     8    18 rmse    standard   1.70      4  0.149  Preprocessor1_Model05
## 10     8    18 rsq     standard   0.589     4  0.0519 Preprocessor1_Model05
## # i 30 more rows
\end{verbatim}

To improve efficiency, parallel computing is used to accelerate
cross-validation and hyperparameter tuning during the model training
process. The tune\_grid() function evaluates multiple combinations of
hyperparameters defined in tune\_RF\_model. After training, the
collect\_metrics() function extracts performance metrics, such as RMSE,
R², and MAE, for each hyperparameter combination.

\begin{Shaded}
\begin{Highlighting}[]
\CommentTok{\# choose the best based off rmse}
\NormalTok{tuned\_RF\_values }\OtherTok{\textless{}{-}} \FunctionTok{select\_best}\NormalTok{(tune\_RF\_results, }\AttributeTok{metric =} \StringTok{"rmse"}\NormalTok{)}
\NormalTok{tuned\_RF\_values}
\end{Highlighting}
\end{Shaded}

\begin{verbatim}
## # A tibble: 1 x 3
##    mtry min_n .config              
##   <int> <int> <chr>                
## 1    15     2 Preprocessor1_Model09
\end{verbatim}

\begin{Shaded}
\begin{Highlighting}[]
\CommentTok{\# specify best combination from tune in workflow}
\NormalTok{RF\_tuned\_wflow }\OtherTok{\textless{}{-}}\NormalTok{RF\_tune\_wflow }\SpecialCharTok{|\textgreater{}}
\NormalTok{  tune}\SpecialCharTok{::}\FunctionTok{finalize\_workflow}\NormalTok{(tuned\_RF\_values)}

\CommentTok{\# fit model with those parameters on train AND test}
\NormalTok{overallfit }\OtherTok{\textless{}{-}}\NormalTok{ RF\_wflow }\SpecialCharTok{|\textgreater{}}
\NormalTok{  tune}\SpecialCharTok{::}\FunctionTok{last\_fit}\NormalTok{(pm\_split)}

\FunctionTok{collect\_metrics}\NormalTok{(overallfit)}
\end{Highlighting}
\end{Shaded}

\begin{verbatim}
## # A tibble: 2 x 4
##   .metric .estimator .estimate .config             
##   <chr>   <chr>          <dbl> <chr>               
## 1 rmse    standard       1.71  Preprocessor1_Model1
## 2 rsq     standard       0.611 Preprocessor1_Model1
\end{verbatim}

\begin{Shaded}
\begin{Highlighting}[]
\NormalTok{test\_predictions }\OtherTok{\textless{}{-}} \FunctionTok{collect\_predictions}\NormalTok{(overallfit)}
\end{Highlighting}
\end{Shaded}

Finally, the select\_best() function identifies the optimal combination
of hyperparameters based on the RMSE metric. The finalized workflow,
updated with the best hyperparameters, is then fitted on the entire
training set and evaluated on the test set. Predicted values for the
test set are retrieved from the final model to assess its performance
and generalizability.

\subsubsection{Extension}\label{extension}

As a reminder, our extension question is: \textbf{Can we find a more
parsimonious model that we can utilize for various situations such as
not having all the variables?}\\
\strut \\
Our extension continues from our analysis. From our analysis, we had
fitted an untuned random forest model to find the top 10 predictors
based on the metric of importance which is measured by the tidymodel's
random forest model. The chart of the top 15 predictors by importance is
shown below.

\begin{Shaded}
\begin{Highlighting}[]
\CommentTok{\# show the top 15 important features given by the built in function by tidymodels}
\NormalTok{RF\_wflow\_fit }\SpecialCharTok{|\textgreater{}} 
  \FunctionTok{extract\_fit\_parsnip}\NormalTok{() }\SpecialCharTok{|\textgreater{}} 
\NormalTok{  vip}\SpecialCharTok{::}\FunctionTok{vip}\NormalTok{(}\AttributeTok{num\_features =} \DecValTok{15}\NormalTok{) }\SpecialCharTok{+}
  \FunctionTok{labs}\NormalTok{(}\AttributeTok{title =} \StringTok{"Top 15 Feature Importance"}\NormalTok{, }\AttributeTok{x =} \StringTok{"Importance"}\NormalTok{) }\SpecialCharTok{+}
  \FunctionTok{aes}\NormalTok{(}\AttributeTok{fill =}\NormalTok{ Importance) }\SpecialCharTok{+} 
  \FunctionTok{scale\_fill\_gradient}\NormalTok{(}\AttributeTok{low =} \StringTok{"slateblue3"}\NormalTok{, }\AttributeTok{high =} \StringTok{"slateblue1"}\NormalTok{)}
\end{Highlighting}
\end{Shaded}

\includegraphics{cs02_files/figure-latex/unnamed-chunk-11-1.pdf}

Now we can create the two random forest models similar to the main model
that include only the top 10 and top 5 predictors.

\begin{Shaded}
\begin{Highlighting}[]
\CommentTok{\# ensure parallelism}
\NormalTok{n\_cores }\OtherTok{\textless{}{-}}\NormalTok{ parallel}\SpecialCharTok{::}\FunctionTok{detectCores}\NormalTok{() }
\NormalTok{doParallel}\SpecialCharTok{::}\FunctionTok{registerDoParallel}\NormalTok{(}\AttributeTok{cores =}\NormalTok{ n\_cores)}

\NormalTok{RF\_t10\_rec }\OtherTok{\textless{}{-}} \FunctionTok{recipe}\NormalTok{(train\_pm) }\SpecialCharTok{|\textgreater{}}
    \CommentTok{\# assign roles for the columns}
    \FunctionTok{update\_role}\NormalTok{(}\FunctionTok{everything}\NormalTok{(), }\AttributeTok{new\_role =} \StringTok{"predictor"}\NormalTok{)}\SpecialCharTok{|\textgreater{}}
    \FunctionTok{update\_role}\NormalTok{(value, }\AttributeTok{new\_role =} \StringTok{"outcome"}\NormalTok{)}\SpecialCharTok{|\textgreater{}}
    \FunctionTok{update\_role}\NormalTok{(id, }\AttributeTok{new\_role =} \StringTok{"id variable"}\NormalTok{) }\SpecialCharTok{|\textgreater{}}
    \FunctionTok{update\_role}\NormalTok{(}\StringTok{"fips"}\NormalTok{, }\AttributeTok{new\_role =} \StringTok{"county id"}\NormalTok{) }\SpecialCharTok{|\textgreater{}}
  
    \CommentTok{\# step\_novel assigns a previously unseen factor level to a new value }
    \FunctionTok{step\_novel}\NormalTok{(}\StringTok{"state"}\NormalTok{) }\SpecialCharTok{|\textgreater{}}
  
    \CommentTok{\# converts string to factors}
    \FunctionTok{step\_string2factor}\NormalTok{(}\StringTok{"state"}\NormalTok{, }\StringTok{"county"}\NormalTok{, }\StringTok{"city"}\NormalTok{) }\SpecialCharTok{|\textgreater{}}

    \CommentTok{\# removing redundancy}
    \FunctionTok{step\_rm}\NormalTok{(}\StringTok{"county"}\NormalTok{) }\SpecialCharTok{|\textgreater{}}
    \FunctionTok{step\_rm}\NormalTok{(}\StringTok{"zcta"}\NormalTok{) }\SpecialCharTok{|\textgreater{}}
  
    \CommentTok{\# removing highly correlated variables}
    \FunctionTok{step\_corr}\NormalTok{(}\FunctionTok{all\_numeric}\NormalTok{()) }\SpecialCharTok{|\textgreater{}}
    
    \CommentTok{\# removing near zero variance predictors}
    \FunctionTok{step\_nzv}\NormalTok{(}\FunctionTok{all\_numeric}\NormalTok{()) }\SpecialCharTok{|\textgreater{}}

    \CommentTok{\# select only top 10 predictors}
    \FunctionTok{step\_select}\NormalTok{(value, state, CMAQ, county\_area, lat, log\_nei\_2008\_pm10\_sum\_10000,}
\NormalTok{                lon, aod, popdens\_county, log\_nei\_2008\_pm10\_sum\_15000,}
\NormalTok{                log\_nei\_2008\_pm10\_sum\_25000)}
    
\CommentTok{\# creating top 10 workflow with model made in analysis}
\NormalTok{RF\_t10\_wflow }\OtherTok{\textless{}{-}}\NormalTok{ workflows}\SpecialCharTok{::}\FunctionTok{workflow}\NormalTok{() }\SpecialCharTok{|\textgreater{}}
\NormalTok{  workflows}\SpecialCharTok{::}\FunctionTok{add\_recipe}\NormalTok{(RF\_t10\_rec) }\SpecialCharTok{|\textgreater{}}
\NormalTok{  workflows}\SpecialCharTok{::}\FunctionTok{add\_model}\NormalTok{(RF\_PM\_model)}

\CommentTok{\# fit with the training data set}
\NormalTok{RF\_t10\_wflow\_fit }\OtherTok{\textless{}{-}}\NormalTok{ parsnip}\SpecialCharTok{::}\FunctionTok{fit}\NormalTok{(RF\_t10\_wflow, }\AttributeTok{data =}\NormalTok{ train\_pm)}

\CommentTok{\# fit with the cross validation }
\FunctionTok{set.seed}\NormalTok{(}\DecValTok{456}\NormalTok{)}
\NormalTok{resample\_t10\_RF\_fit }\OtherTok{\textless{}{-}}\NormalTok{ tune}\SpecialCharTok{::}\FunctionTok{fit\_resamples}\NormalTok{(RF\_t10\_wflow, vfold\_pm)}
\FunctionTok{collect\_metrics}\NormalTok{(resample\_t10\_RF\_fit)}
\end{Highlighting}
\end{Shaded}

\begin{verbatim}
## # A tibble: 2 x 6
##   .metric .estimator  mean     n std_err .config             
##   <chr>   <chr>      <dbl> <int>   <dbl> <chr>               
## 1 rmse    standard   1.65      4  0.136  Preprocessor1_Model1
## 2 rsq     standard   0.581     4  0.0481 Preprocessor1_Model1
\end{verbatim}

\begin{Shaded}
\begin{Highlighting}[]
\CommentTok{\# initializing a tuned random forest model}
\NormalTok{tune\_RF\_t10\_model }\OtherTok{\textless{}{-}} \FunctionTok{rand\_forest}\NormalTok{(}\AttributeTok{mtry =} \FunctionTok{tune}\NormalTok{(), }\AttributeTok{min\_n =} \FunctionTok{tune}\NormalTok{()) }\SpecialCharTok{|\textgreater{}}
  \FunctionTok{set\_engine}\NormalTok{(}\StringTok{"randomForest"}\NormalTok{) }\SpecialCharTok{|\textgreater{}}
  \FunctionTok{set\_mode}\NormalTok{(}\StringTok{"regression"}\NormalTok{)}

\CommentTok{\# create workflow}
\NormalTok{RF\_t10\_tune\_wflow }\OtherTok{\textless{}{-}}\NormalTok{ workflows}\SpecialCharTok{::}\FunctionTok{workflow}\NormalTok{() }\SpecialCharTok{|\textgreater{}}
\NormalTok{  workflows}\SpecialCharTok{::}\FunctionTok{add\_recipe}\NormalTok{(RF\_t10\_rec) }\SpecialCharTok{|\textgreater{}}
\NormalTok{  workflows}\SpecialCharTok{::}\FunctionTok{add\_model}\NormalTok{(tune\_RF\_t10\_model)}

\CommentTok{\# perform hyper parameter tuning for top 10 random forest model using grid search across re samples}
\FunctionTok{set.seed}\NormalTok{(}\DecValTok{123}\NormalTok{)}
\NormalTok{tune\_RF\_t10\_results }\OtherTok{\textless{}{-}} \FunctionTok{tune\_grid}\NormalTok{(}\AttributeTok{object =}\NormalTok{ RF\_t10\_tune\_wflow, }\AttributeTok{resamples =}\NormalTok{ vfold\_pm, }\AttributeTok{grid =} \DecValTok{20}\NormalTok{)}

\CommentTok{\# results}
\NormalTok{tune\_RF\_t10\_results }\SpecialCharTok{|\textgreater{}}
  \FunctionTok{collect\_metrics}\NormalTok{()}
\end{Highlighting}
\end{Shaded}

\begin{verbatim}
## # A tibble: 40 x 8
##     mtry min_n .metric .estimator  mean     n std_err .config              
##    <int> <int> <chr>   <chr>      <dbl> <int>   <dbl> <chr>                
##  1     1    14 rmse    standard   1.73      4  0.147  Preprocessor1_Model01
##  2     1    14 rsq     standard   0.575     4  0.0527 Preprocessor1_Model01
##  3     1    24 rmse    standard   1.76      4  0.142  Preprocessor1_Model02
##  4     1    24 rsq     standard   0.563     4  0.0489 Preprocessor1_Model02
##  5     2    34 rmse    standard   1.71      4  0.151  Preprocessor1_Model03
##  6     2    34 rsq     standard   0.575     4  0.0542 Preprocessor1_Model03
##  7     2     6 rmse    standard   1.63      4  0.139  Preprocessor1_Model04
##  8     2     6 rsq     standard   0.610     4  0.0431 Preprocessor1_Model04
##  9     3    18 rmse    standard   1.65      4  0.146  Preprocessor1_Model05
## 10     3    18 rsq     standard   0.595     4  0.0488 Preprocessor1_Model05
## # i 30 more rows
\end{verbatim}

\begin{Shaded}
\begin{Highlighting}[]
\CommentTok{\# select best model based on rmse}
\NormalTok{tuned\_RF\_t10\_values}\OtherTok{\textless{}{-}} \FunctionTok{select\_best}\NormalTok{(tune\_RF\_t10\_results, }\AttributeTok{metric =} \StringTok{"rmse"}\NormalTok{)}

\CommentTok{\# specify best combination from tune in workflow}
\NormalTok{RF\_t10\_tuned\_wflow }\OtherTok{\textless{}{-}}\NormalTok{RF\_t10\_tune\_wflow }\SpecialCharTok{|\textgreater{}}
\NormalTok{  tune}\SpecialCharTok{::}\FunctionTok{finalize\_workflow}\NormalTok{(tuned\_RF\_t10\_values)}

\CommentTok{\# fit model with those parameters on train AND test}
\NormalTok{overallfit\_t10 }\OtherTok{\textless{}{-}}\NormalTok{ RF\_t10\_wflow }\SpecialCharTok{|\textgreater{}}
\NormalTok{  tune}\SpecialCharTok{::}\FunctionTok{last\_fit}\NormalTok{(pm\_split)}

\NormalTok{test\_t10\_predictions }\OtherTok{\textless{}{-}} \FunctionTok{collect\_predictions}\NormalTok{(overallfit)}

\CommentTok{\# build a recipe with only top 5 features}
\NormalTok{RF\_t5\_rec }\OtherTok{\textless{}{-}} \FunctionTok{recipe}\NormalTok{(train\_pm) }\SpecialCharTok{|\textgreater{}}
    \CommentTok{\# assign roles for the columns}
    \FunctionTok{update\_role}\NormalTok{(}\FunctionTok{everything}\NormalTok{(), }\AttributeTok{new\_role =} \StringTok{"predictor"}\NormalTok{)}\SpecialCharTok{|\textgreater{}}
    \FunctionTok{update\_role}\NormalTok{(value, }\AttributeTok{new\_role =} \StringTok{"outcome"}\NormalTok{)}\SpecialCharTok{|\textgreater{}}
    \FunctionTok{update\_role}\NormalTok{(id, }\AttributeTok{new\_role =} \StringTok{"id variable"}\NormalTok{) }\SpecialCharTok{|\textgreater{}}
    \FunctionTok{update\_role}\NormalTok{(}\StringTok{"fips"}\NormalTok{, }\AttributeTok{new\_role =} \StringTok{"county id"}\NormalTok{) }\SpecialCharTok{|\textgreater{}}
    
    \CommentTok{\# step\_novel assigns a previously unseen factor level to a new value }
    \FunctionTok{step\_novel}\NormalTok{(}\StringTok{"state"}\NormalTok{) }\SpecialCharTok{|\textgreater{}}
    
    \CommentTok{\# converts string to factors}
    \FunctionTok{step\_string2factor}\NormalTok{(}\StringTok{"state"}\NormalTok{, }\StringTok{"county"}\NormalTok{, }\StringTok{"city"}\NormalTok{) }\SpecialCharTok{|\textgreater{}}
  
    \CommentTok{\# removing redundancy}
    \FunctionTok{step\_rm}\NormalTok{(}\StringTok{"county"}\NormalTok{) }\SpecialCharTok{|\textgreater{}}
    \FunctionTok{step\_rm}\NormalTok{(}\StringTok{"zcta"}\NormalTok{) }\SpecialCharTok{|\textgreater{}}
    \CommentTok{\# removing highly correlated variables}
    \FunctionTok{step\_corr}\NormalTok{(}\FunctionTok{all\_numeric}\NormalTok{())}\SpecialCharTok{|\textgreater{}}
    \CommentTok{\# removing near zero variance predictors}
    \FunctionTok{step\_nzv}\NormalTok{(}\FunctionTok{all\_numeric}\NormalTok{()) }\SpecialCharTok{|\textgreater{}}
    
    \CommentTok{\# select only top 5 predictors}
    \FunctionTok{step\_select}\NormalTok{(value, state, CMAQ, county\_area, lat, log\_nei\_2008\_pm10\_sum\_10000)}

\CommentTok{\# creating top 5 workflow with model made in analysis}
\NormalTok{RF\_t5\_wflow }\OtherTok{\textless{}{-}}\NormalTok{ workflows}\SpecialCharTok{::}\FunctionTok{workflow}\NormalTok{() }\SpecialCharTok{|\textgreater{}}
\NormalTok{  workflows}\SpecialCharTok{::}\FunctionTok{add\_recipe}\NormalTok{(RF\_t5\_rec) }\SpecialCharTok{|\textgreater{}}
\NormalTok{  workflows}\SpecialCharTok{::}\FunctionTok{add\_model}\NormalTok{(RF\_PM\_model)}

\CommentTok{\# fit with the training data set}
\NormalTok{RF\_t5\_wflow\_fit }\OtherTok{\textless{}{-}}\NormalTok{ parsnip}\SpecialCharTok{::}\FunctionTok{fit}\NormalTok{(RF\_t5\_wflow, }\AttributeTok{data =}\NormalTok{ train\_pm)}

\CommentTok{\# fit with the cross validation }
\FunctionTok{set.seed}\NormalTok{(}\DecValTok{456}\NormalTok{)}
\NormalTok{resample\_t5\_RF\_fit }\OtherTok{\textless{}{-}}\NormalTok{ tune}\SpecialCharTok{::}\FunctionTok{fit\_resamples}\NormalTok{(RF\_t5\_wflow, vfold\_pm)}
\FunctionTok{collect\_metrics}\NormalTok{(resample\_t5\_RF\_fit)}
\end{Highlighting}
\end{Shaded}

\begin{verbatim}
## # A tibble: 2 x 6
##   .metric .estimator  mean     n std_err .config             
##   <chr>   <chr>      <dbl> <int>   <dbl> <chr>               
## 1 rmse    standard   1.65      4  0.135  Preprocessor1_Model1
## 2 rsq     standard   0.581     4  0.0518 Preprocessor1_Model1
\end{verbatim}

\begin{Shaded}
\begin{Highlighting}[]
\CommentTok{\# initialize a tuned random forest model}
\NormalTok{tune\_RF\_t5\_model }\OtherTok{\textless{}{-}} \FunctionTok{rand\_forest}\NormalTok{(}\AttributeTok{mtry =} \FunctionTok{tune}\NormalTok{(), }\AttributeTok{min\_n =} \FunctionTok{tune}\NormalTok{()) }\SpecialCharTok{|\textgreater{}}
  \FunctionTok{set\_engine}\NormalTok{(}\StringTok{"randomForest"}\NormalTok{) }\SpecialCharTok{|\textgreater{}}
  \FunctionTok{set\_mode}\NormalTok{(}\StringTok{"regression"}\NormalTok{)}

\CommentTok{\# create workflow}
\NormalTok{RF\_t5\_tune\_wflow }\OtherTok{\textless{}{-}}\NormalTok{ workflows}\SpecialCharTok{::}\FunctionTok{workflow}\NormalTok{() }\SpecialCharTok{|\textgreater{}}
\NormalTok{  workflows}\SpecialCharTok{::}\FunctionTok{add\_recipe}\NormalTok{(RF\_t5\_rec) }\SpecialCharTok{|\textgreater{}}
\NormalTok{  workflows}\SpecialCharTok{::}\FunctionTok{add\_model}\NormalTok{(tune\_RF\_t5\_model)}

\CommentTok{\# perform hyper parameter tuning for top 5 random forest model using grid search across re samples}
\FunctionTok{set.seed}\NormalTok{(}\DecValTok{123}\NormalTok{)}
\NormalTok{tune\_RF\_t5\_results }\OtherTok{\textless{}{-}} \FunctionTok{tune\_grid}\NormalTok{(}\AttributeTok{object =}\NormalTok{ RF\_t5\_tune\_wflow, }\AttributeTok{resamples =}\NormalTok{ vfold\_pm, }\AttributeTok{grid =} \DecValTok{20}\NormalTok{)}

\CommentTok{\# results}
\NormalTok{tune\_RF\_t5\_results }\SpecialCharTok{|\textgreater{}}
  \FunctionTok{collect\_metrics}\NormalTok{()}
\end{Highlighting}
\end{Shaded}

\begin{verbatim}
## # A tibble: 40 x 8
##     mtry min_n .metric .estimator  mean     n std_err .config              
##    <int> <int> <chr>   <chr>      <dbl> <int>   <dbl> <chr>                
##  1     1    14 rmse    standard   1.72      4  0.131  Preprocessor1_Model01
##  2     1    14 rsq     standard   0.570     4  0.0455 Preprocessor1_Model01
##  3     1    24 rmse    standard   1.74      4  0.135  Preprocessor1_Model02
##  4     1    24 rsq     standard   0.561     4  0.0470 Preprocessor1_Model02
##  5     1    34 rmse    standard   1.77      4  0.139  Preprocessor1_Model03
##  6     1    34 rsq     standard   0.546     4  0.0484 Preprocessor1_Model03
##  7     1     6 rmse    standard   1.69      4  0.135  Preprocessor1_Model04
##  8     1     6 rsq     standard   0.583     4  0.0450 Preprocessor1_Model04
##  9     2    18 rmse    standard   1.67      4  0.137  Preprocessor1_Model05
## 10     2    18 rsq     standard   0.579     4  0.0503 Preprocessor1_Model05
## # i 30 more rows
\end{verbatim}

\begin{Shaded}
\begin{Highlighting}[]
\CommentTok{\# select best model based on rmse}
\NormalTok{tuned\_RF\_t5\_values}\OtherTok{\textless{}{-}} \FunctionTok{select\_best}\NormalTok{(tune\_RF\_t5\_results, }\AttributeTok{metric =} \StringTok{"rmse"}\NormalTok{)}

\CommentTok{\# specify best combination from tune in workflow}
\NormalTok{RF\_t5\_tuned\_wflow }\OtherTok{\textless{}{-}}\NormalTok{RF\_t5\_tune\_wflow }\SpecialCharTok{|\textgreater{}}
\NormalTok{  tune}\SpecialCharTok{::}\FunctionTok{finalize\_workflow}\NormalTok{(tuned\_RF\_t5\_values)}

\CommentTok{\# fit model with those parameters on train AND test}
\NormalTok{overallfit\_t5 }\OtherTok{\textless{}{-}}\NormalTok{ RF\_t5\_wflow }\SpecialCharTok{|\textgreater{}}
\NormalTok{  tune}\SpecialCharTok{::}\FunctionTok{last\_fit}\NormalTok{(pm\_split)}

\NormalTok{test\_t5\_predictions }\OtherTok{\textless{}{-}} \FunctionTok{collect\_predictions}\NormalTok{(overallfit)}
\end{Highlighting}
\end{Shaded}

\subsection{Results \& Discussion}\label{results-discussion}

For our results, we will be using the following metrics:

\begin{itemize}
\tightlist
\item
  \(adj. r^2\): The adjusted r-squared value, how much a model
  corresponds to the variance of the predicted values, adjusted to
  account for the amount of predictors. We use this value to compare the
  models on a scale of accuracy while also taking into consideration
  that less features are better. This is a good measure for how
  parsimonious a model is.
\item
  \(rmse\): Root mean squared error, the average error between all
  predicted and truth values, a metric of distance.
\end{itemize}

We ran our random forest models on the whole data set to get the
following tables:

\begin{verbatim}
## overall model
\end{verbatim}

\begin{verbatim}
## # A tibble: 2 x 4
##   .metric .estimator .estimate .config             
##   <chr>   <chr>          <dbl> <chr>               
## 1 rmse    standard       1.71  Preprocessor1_Model1
## 2 rsq     standard       0.611 Preprocessor1_Model1
\end{verbatim}

\begin{verbatim}
## top10 model
\end{verbatim}

\begin{verbatim}
## # A tibble: 2 x 4
##   .metric .estimator .estimate .config             
##   <chr>   <chr>          <dbl> <chr>               
## 1 rmse    standard       1.84  Preprocessor1_Model1
## 2 rsq     standard       0.532 Preprocessor1_Model1
\end{verbatim}

\begin{verbatim}
## top5 model
\end{verbatim}

\begin{verbatim}
## # A tibble: 2 x 4
##   .metric .estimator .estimate .config             
##   <chr>   <chr>          <dbl> <chr>               
## 1 rmse    standard       1.93  Preprocessor1_Model1
## 2 rsq     standard       0.497 Preprocessor1_Model1
\end{verbatim}

However, there isn't a built-in measurement for \(adj. r^2\) in
tidyverse so we would need to create a function to do it ourselves:

\begin{Shaded}
\begin{Highlighting}[]
\CommentTok{\# function for calculating adjusted r squared (chatgpt generated)}
\NormalTok{calculate\_adj\_r\_squared\_tidy }\OtherTok{\textless{}{-}} \ControlFlowTok{function}\NormalTok{(predictions, truth\_col, estimate\_col, num\_predictors) \{}
  
  \CommentTok{\# Compute R{-}squared}
\NormalTok{  rsq }\OtherTok{\textless{}{-}} \FunctionTok{rsq\_vec}\NormalTok{(truth\_col, estimate\_col)}
  
  \CommentTok{\# Number of observations}
\NormalTok{  n }\OtherTok{\textless{}{-}} \FunctionTok{nrow}\NormalTok{(predictions)}
  
  \CommentTok{\# Number of predictors}
\NormalTok{  p }\OtherTok{\textless{}{-}}\NormalTok{ num\_predictors}
  
  \CommentTok{\# Compute Adjusted R{-}squared}
\NormalTok{  adj\_r\_squared }\OtherTok{\textless{}{-}} \DecValTok{1} \SpecialCharTok{{-}}\NormalTok{ ((}\DecValTok{1} \SpecialCharTok{{-}}\NormalTok{ rsq) }\SpecialCharTok{*}\NormalTok{ (n }\SpecialCharTok{{-}} \DecValTok{1}\NormalTok{) }\SpecialCharTok{/}\NormalTok{ (n }\SpecialCharTok{{-}}\NormalTok{ p }\SpecialCharTok{{-}} \DecValTok{1}\NormalTok{))}
  
  \FunctionTok{return}\NormalTok{(adj\_r\_squared)}
\NormalTok{\}}


\CommentTok{\# we must get the number of predictors for each model using a function (chatgpt generated)}
\NormalTok{get\_predictor\_details }\OtherTok{\textless{}{-}} \ControlFlowTok{function}\NormalTok{(workflow\_fit) \{}
  \CommentTok{\# Extract the final model fit from the workflow}
\NormalTok{  final\_model }\OtherTok{\textless{}{-}}\NormalTok{ workflows}\SpecialCharTok{::}\FunctionTok{extract\_fit\_parsnip}\NormalTok{(workflow\_fit)}
  
  \CommentTok{\# Extract predictor importance}
\NormalTok{  predictor\_importance }\OtherTok{\textless{}{-}}\NormalTok{ final\_model}\SpecialCharTok{$}\NormalTok{fit}\SpecialCharTok{$}\NormalTok{importance}
  
  \CommentTok{\# Get the names of the predictors}
\NormalTok{  predictor\_names }\OtherTok{\textless{}{-}} \FunctionTok{rownames}\NormalTok{(predictor\_importance)}
  
  \CommentTok{\# Check the number of predictors}
\NormalTok{  num\_predictors }\OtherTok{\textless{}{-}} \FunctionTok{length}\NormalTok{(predictor\_names)}
  
  \CommentTok{\# Return results as a list}
  \FunctionTok{list}\NormalTok{(}
    \AttributeTok{num\_predictors =}\NormalTok{ num\_predictors,}
    \AttributeTok{predictor\_names =}\NormalTok{ predictor\_names,}
    \AttributeTok{predictor\_importance =}\NormalTok{ predictor\_importance}
\NormalTok{  )}
\NormalTok{\}}

\CommentTok{\# Collect predictions from the fitted t10 model}
\NormalTok{test\_t10\_predictions }\OtherTok{\textless{}{-}} \FunctionTok{collect\_predictions}\NormalTok{(overallfit\_t10)}
\CommentTok{\# Get the number of predictors of the model using a function}
\NormalTok{num\_predictors\_t10 }\OtherTok{\textless{}{-}} \FunctionTok{get\_predictor\_details}\NormalTok{(overallfit\_t10)}\SpecialCharTok{$}\NormalTok{num\_predictors}
\CommentTok{\# Use the function to calculate adj r\^{}2}
\NormalTok{adj\_r\_squared\_t10 }\OtherTok{\textless{}{-}} \FunctionTok{calculate\_adj\_r\_squared\_tidy}\NormalTok{(}
  \AttributeTok{predictions =}\NormalTok{ test\_t10\_predictions,}
  \AttributeTok{truth\_col =}\NormalTok{ test\_t10\_predictions}\SpecialCharTok{$}\NormalTok{value,}
  \AttributeTok{estimate\_col =}\NormalTok{ test\_t10\_predictions}\SpecialCharTok{$}\NormalTok{.pred,}
  \AttributeTok{num\_predictors =}\NormalTok{ num\_predictors\_t10}
\NormalTok{)}

\CommentTok{\# Collect predictions from the fitted t5 model}
\NormalTok{test\_t5\_predictions }\OtherTok{\textless{}{-}} \FunctionTok{collect\_predictions}\NormalTok{(overallfit\_t5)}
\CommentTok{\# Get the number of predictors of the model using a function}
\NormalTok{num\_predictors\_t5 }\OtherTok{\textless{}{-}} \FunctionTok{get\_predictor\_details}\NormalTok{(overallfit\_t5)}\SpecialCharTok{$}\NormalTok{num\_predictors}
\CommentTok{\# Use the function to calculate adj r\^{}2}
\NormalTok{adj\_r\_squared\_t5 }\OtherTok{\textless{}{-}} \FunctionTok{calculate\_adj\_r\_squared\_tidy}\NormalTok{(}
  \AttributeTok{predictions =}\NormalTok{ test\_t5\_predictions,}
  \AttributeTok{truth\_col =}\NormalTok{ test\_t5\_predictions}\SpecialCharTok{$}\NormalTok{value,}
  \AttributeTok{estimate\_col =}\NormalTok{ test\_t5\_predictions}\SpecialCharTok{$}\NormalTok{.pred,}
  \AttributeTok{num\_predictors =}\NormalTok{ num\_predictors\_t10}
\NormalTok{)}

\CommentTok{\# Collect predictions from the fitted overall model}
\NormalTok{test\_overall\_predictions }\OtherTok{\textless{}{-}} \FunctionTok{collect\_predictions}\NormalTok{(overallfit)}
\CommentTok{\# Get the number of predictors of the model using a function}
\NormalTok{num\_predictors\_overall }\OtherTok{\textless{}{-}} \FunctionTok{get\_predictor\_details}\NormalTok{(overallfit)}\SpecialCharTok{$}\NormalTok{num\_predictors}
\CommentTok{\# Use the function to calculate adj r\^{}2}
\NormalTok{adj\_r\_squared\_overall }\OtherTok{\textless{}{-}} \FunctionTok{calculate\_adj\_r\_squared\_tidy}\NormalTok{(}
  \AttributeTok{predictions =}\NormalTok{ test\_overall\_predictions,}
  \AttributeTok{truth\_col =}\NormalTok{ test\_overall\_predictions}\SpecialCharTok{$}\NormalTok{value,}
  \AttributeTok{estimate\_col =}\NormalTok{ test\_overall\_predictions}\SpecialCharTok{$}\NormalTok{.pred,}
  \AttributeTok{num\_predictors =}\NormalTok{ num\_predictors\_t10}
\NormalTok{)}
\end{Highlighting}
\end{Shaded}

\begin{verbatim}
## overall adj. r^2
\end{verbatim}

\begin{verbatim}
## [1] 0.5966578
\end{verbatim}

\begin{verbatim}
## t10 adj. r^2
\end{verbatim}

\begin{verbatim}
## [1] 0.5157635
\end{verbatim}

\begin{verbatim}
## t5 adj. r^2
\end{verbatim}

\begin{verbatim}
## [1] 0.4787824
\end{verbatim}

Our results here are what we expected. The less features we use, the
less accurate our models are when predicting for the \texttt{value}.
Here's the breakdown for each model:

\begin{itemize}
\tightlist
\item
  \texttt{overall} baseline model: The overall random forest model that
  include all the predictors has the lowest \(RMSE\) of 1.72 and highest
  \(adj. R^2\) of 0.594. This would mean that the baseline model has the
  best accuracy and explains for about 59.4\% of the variance in the
  target variable \texttt{value}. This will be our baseline model.
\item
  \texttt{top10} model: The model with top 10 important features had its
  \(RMSE\) slightly increase to 1.85, indicating a slightly reduced
  accuracy. The model's \(adj. R^2\) dropped to 0.510, which shows that
  less variance of \texttt{value} (51\%) is explained by the model.
\item
  \texttt{top5} models: The model with top 5 important features dropped
  in acurracy even more than the \texttt{top10} model to a \texttt{RMSE}
  of 1.92. We also saw the \(adj. R^2\) drop to 0.479, meaning the model
  explains about 48\% of the variance of \texttt{value}.
\end{itemize}

As we decrease the amount of features, there is a decrease in accuracy
and \(adj. R^2\). This trade-off may be acceptable in some cases where
model simplicity is needed where monitors only collect a select amount
of variables. To explicitly answer the questions, we can reasonably
predict the US annual average air pollution concentration with
reasonable accuracy of 1.72 \(RMSE\) and 0.594 \(adj. R^2\) using our
baseline model, explaining 59.4\% of the variance in air pollution
levels. The top-10 and top-5 models demonstrates a trade-off that a
simpler model requires sacrificing some accuracy and explanatory power.
For the most accurate and reliable prediction, the baseline model
remains the best choice. However, if simpler approach is needed, both
the top 10 and top 5 models offer practical approaches with reasonable
accuracies, with the top 5 being slightly less accurate than the top 10
model (\texttt{RMSE}: 1.92 vs 1.85).

\subsection{Conclusion}\label{conclusion}

Through the random forest model, our team demonstrated we can predict
the U.S. annual average air pollution concentration with an average
error of 1.72, and find a more parsimonious model through feature
selection with average errors of 1.85 and 1.92; these models can raise
awareness and detect these undetectable threats in the U.S. As stated
above, we conclude that the parsimonious model with top ten best
features is comparable to our overall model, an approximate 0.1 error
difference. Therefore, if an individual cannot gather all sixteen
predictors to predict the air pollution concentration, our top ten model
is more practical and flexible to utilize with less predictors to
measure.

\subsection{Limitations}\label{limitations}

Although our model may provide a general accuracy in predicting the
concentration, there are still several factors to consider when using
our model as a representation. It is important to consider that the data
used for this study is a decade old and solely measured in the U.S.. Air
pollution readings may fluctuate throughout the years and may serve our
model as ineffective for future readings. Recent data can be analyzed
and implemented into our model for more accurate predictions. Adding on,
as the monitor measurements are solely in the U.S., this is not a
universal model that we can utilize for other parts of the world. Stated
previously, it is also important to consider that our data does not
include specific details on the composition of particulate matter. There
are several types of particulates that should be considered as they may
have a range of effects on human health. We worked on this project
through a scientific lens and did not focus on factors from other
perspectives. Our analysis is derived from data focused on major biomes
in the United States (e.g., states, cities, and counties) and excludes
the smaller regions (microbiomes) within. History of racism throughout
multiple parts of the United States have made some of these microbiomes
- usually non-white neighborhoods - have significantly higher indexes of
air pollution compared to white neighborhoods. For instance, in Los
Angeles, infrastructure projects were intentionally built through Black
and Latino neighborhoods often displacing families and subjecting the
people within these communities to higher exposure to pollutants
\footnote{\url{https://afrolanews.org/2023/06/unequal-air-the-pollution-legacy-of-segregation-and-the-freeway-boom-in-los-angeles/}}.
As a result, grouping several regions within a city may greatly skew the
calculation of the annual average and misrepresent the region's general
air quality. Lastly, we mainly focused on outdoor exposure to pollution
and did not take into account individual exposure in private
environments and other individual activity. It is said that air
fresheners, gas stoves, and exposure to tobacco smoke can also
negatively impact one's health in the long term \footnote{\url{https://www.epa.gov/indoor-air-quality-iaq/inside-story-guide-indoor-air-quality\#}:\textasciitilde:text=Some\%20sources\%2C\%20such\%20as\%20building,after\%20some\%20of\%20these\%20activities.}.

\end{document}
